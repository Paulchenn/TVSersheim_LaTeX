\documentclass[
% Schriftgröße
fontsize=12pt,
%
% zwischen Absätzen eine leere Zeile einfügen, statt lediglich Einrückung
parskip=full,
%
% Papierformat auf DIN-A4
paper=A4,
%
% Briefkopf (ganz oben) rechts ausrichten, standardmäßig links
fromalign=right,
%
% Telefonnummer im Briefkopf anzeigen
fromphone=true,
%
% Faxnnummer im Briefkopf anzeigen
%fromfax=true,
%
% E-Mail-Adresse im Briefkopf anzeigen
fromemail=true,
%
% URL im Briefkopf anzeigen
%fromurl=true,
%
% Faltmarkierungen verbergen
%foldmarks=false,
%
% Die neuste Version von scrlettr2 verwenden
version=last,
]{scrlttr2}

% Zeichenkodierung des Dokuments ist in UTF-8
\usepackage[utf8]{inputenc}
% Helvetica (ähnlich Arial) laden
\usepackage[scaled]{helvet}

% Eurosymbol-Unterstützung
\usepackage{eurosym}
% Das Unicode-Zeichen € als \euro interpretieren.
% So kann man direkt € tippen anstatt jedes Mal \euro auszuschreiben.
\DeclareUnicodeCharacter{20AC}{\euro}

% Sprache des Dokuments auf Deutsch
\usepackage[ngerman]{babel}

% Includen von PDFs nach dem Brief, siehe \includepdf unten
\usepackage{pdfpages}

% klickbare Links und E-Mail-Adressen. Paket url kann keine klickbaren,
% deswegen hyperref. Option hidelinks versteckt farbigen Rahmen.
\usepackage[colorlinks=true, linkcolor=blue, urlcolor=blue,]{hyperref}

% Für Schlussformel (und nicht vorhandenen Namen darunter) Linksbündigkeit erzwingen
%\renewcommand*{\raggedsignature}{\raggedright}

% Paket zum Platzieren des Bildes im Hintergrund
\usepackage{eso-pic}
% Paket zum Einfügen von Bildern
\usepackage{graphicx}
% Paket zur Anpassung der Deckkraft
\usepackage{transparent}

% Zum Erzeugen von Blindtext
\usepackage{lipsum}
\usepackage{titlesec}
%\usepackage[a4paper, left=1.7cm, right=4cm, top=2cm, bottom=2.5cm]{geometry}

% Absendername unter Schlussformel entfernen. Dieser wird bereits aus dem Briefkopf ersichtlich.
% Hier wird die signature-Variable einfach auf einen leeren Wert gesetzt und wäre sonst \usekomavar{fromname}.
\setkomavar{signature}{}

\renewcommand{\familydefault}{\sfdefault}  % Serifenlose Schrift als Standard setzen
\setlength{\parindent}{0pt}  % Setzt den Einzug der ersten Zeile auf 0pt



\begin{document}
	
	% Füge die PDF-Datei als Hintergrundbild hinzu
	\AddToShipoutPictureBG*{
		\includegraphics[width=\paperwidth,height=\paperheight]{Hintergrund_BriefFooter_120J.pdf}
	}
	
	% Datum
	\setkomavar{date}{\today}
	
	% Betreff
	\setkomavar{subject}{Auslage für TV Sersheim 1904 e.V.}
	
	% Kundennummer
	%\setkomavar{customer}[\customername]{DE-112233}
	
	% Ihr Zeichen
	%\setkomavar{yourref}[\yourrefname]{IZ-12345}
	
	% Ihr Schreiben vom
	%\setkomavar{yourmail}[\yourmailname]{11. Juli 2024 (über Amazon)}
	
	\begin{letter}{
			Max Mustermann\\
			Musterstraße 23\\
			74372 Sersheim
		}
		
		\opening{Hallo Max,}
		
		
		\lipsum[1]
		
		
		\closing{Mit freundlichen Grüßen}
		Paul Glaser
		
		% Post Scriptum
		%\ps PS: Ich bin bis März nur telefonisch erreichbar.
		
		% Anlage(n)
		% Standardmäßig wird "Anlage(n)" eingefügt, dies kann überschrieben werden, hier mit "Anlagen"
		%\setkomavar*{enclseparator}{Anlagen}
		%\encl{Kopie des Ausweises}
		
		% Verteiler
		%\cc{Bürgermeister, Vereinsvorsitzender}
		
	\end{letter}
	
	% Weitere PDFs können automatisch angefügt werden, z.B. Ahnänge.
	%\includepdf[pages=-,openright]{pfad/zu/weiteren/pdfs/dokument.pdf}
	% Pfad ist relativ zu dieser tex-Datei. Mit .. ein Verzeichnis hoch.
	% Der pages-Parameter spezifiziert welche Seiten eingefügt werden.
	% Beispiele:
	% pages=-				alle Seiten
	% pages={1-4}			Seite 1-4
	% pages={1,4,5}			Seite 1, 4 und 5
	% pages={3,{},8-11,15}	Seite 3, leere Seite, Seite 8-11 und Seite 15
	% Der openright-Parameter startet die Anlagen auf ungerader (rechter) Seite, d.h. notfalls wird eine leere Seite
	% eingefügt. Im doppelseitigem Druck wird dadurch besser zwischen Brief und Anlage getrennt. Für einseitigen Druck
	% entfernen.
	
	
\end{document}
